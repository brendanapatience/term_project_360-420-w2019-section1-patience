\documentclass[a4paper,12pt]{article}
\usepackage[utf8]{inputenc}
\usepackage{indentfirst}
\usepackage{pgfplots}

\title{The Double Pendulum}
\author{Brendan Patience}

\begin{document}

\maketitle

\section*{Introduction}

My project was to simulate and analyze the double pendulum. Since this kind
of pendulum is a chaotic system, I wanted to see how two different numerical
methods would calculate the path. I chose Euler's method and the fourth
order Runge-Kutta method. In fact, Euler's method was just recently taught
and was explained to be one  of the easiest methods to numerically integrate
ordinary differential equations. But due to its simplicity, it also has a
higher degree of error. I wanted to see how it would perform in a chaotic
system. To determine whether it is indeed a viable solution for the equations
of motion of the double pendulum, I analyzed the mechanical energy of the
system. The only force acting on the pendulum in my simulation is gravity, 
so the total energy must be conserved. My hypothesis was that Euler's method
would be too simplistic as an ODE solver and that this would manifest in
a change of mechanical energy (whether decrease or increase, but not stay
constant).

\section*{Description of Numerical Method}

Before applying the ODE solvers, the equations of motion of the pendulum
must be obtained. The first step is to find the Lagrangian of the 
pendulum, given by

\[L = T - V \]

where T is the kinetic energy and V is the potential energy. We can then
obtain the canonical momentum equations, by deriving the Lagrangian with
respect to the angular velocity of each bob:

\[p_1 = \frac{\partial L}{\partial \dot{\theta_1}} \]
\[p_2 = \frac{\partial L}{\partial \dot{\theta_2}} \]

The next step is to convert the Lagrangian equation into the Hamiltonian
equation, expressed by the following relationship

\[H = \sum_{i=1}^{2} \left( \dot{\theta_i} p_i \right) - L \]

Finally, the Hamilton equations can be obtained by deriving the Hamiltonian
with respect to both momenta and both angles.

\[ \dot{\theta_i} = \frac{\partial H}{\partial p_i} \]
\[ \dot{p_i} = - \frac{\partial H}{\partial \theta_i} \]

The final equations of motion are the following:

\[ \dot{\theta_1} = \frac{l p_1 - l p_2 \cos \left(\theta_1 - \theta_2 \right)}
						 {l^3 \left[ m_1 + m_2 \sin^2 \left( \theta_1 - 
							\theta_2 \right) \right]} \]

\[ \dot{\theta_2} = \frac{l \left(m_1 + m_2\right) p_2 - l m_2 p_1 \cos 
									\left(\theta_1 - \theta_2 \right)}
						 {l^3 m_2 \left[ m_1 + m_2 \sin^2 \left( \theta_1 - 
							\theta_2 \right) \right]} \]

\[ \dot{p_1} = - \left(m_1 + m_2\right) g l \sin \theta_1 - A_1 + A_2\]

\[ \dot{p_2} = - m_2 g l \sin \theta_2 + A_1 - A_2\]

where

\[ A_1 = \frac{p_1 p_2 \sin \left( \theta_1 - \theta_2 \right)}
			  {l^2 \left[m_1 + m_2 sin^2 \left( \theta_1 - \theta_2 \right)\right]} \]
			  
\[ A_2 = \frac{l^2 m_2 p_1^2 + l^2 \left( m_1 + m_2 \right) p_2^2 - l^2 m_2 p_1 p_2 
					\cos \left( \theta_1 - \theta_2 \right)}
			  {2 l^4 \left[ m_1 + m_2 \sin^2 \left( \theta_1 - 
					\theta_2 \right)\right]^2}
		 \sin \left[ 2 \right( \theta_1 - \theta_2 \left) \right] \]

At this point there are four first order differential equations. These can
now be solved using the numerical methods in question, namely Euler's and
the fourth order Runge-Kutta. \\ \\
\indent Euler's:

\[ y_{n+1} = y_n + h f' \left( t_n, y_n \right)\]

RK4:

\[ k_1 = h f' \left(t_n, y_n \right) \]
\[ k_2 = h f' \left(t_n + \frac{h}{2}, y_n + \frac{k_1}{2} \right) \]
\[ k_3 = h f' \left(t_n + \frac{h}{2}, y_n + \frac{k_1}{2} \right) \]
\[ k_4 = h f' \left(t_n + h, y_n + k_3 \right) \]

\[ y_{n+1} = y_n + \frac{k_1}{6} + \frac{k_2}{3} + \frac{k_3}{3} + 
					\frac{k_4}{6} + O \left( h^5 \right)\]

By plugging the values into the energy equations we can obtain the total
energy:

\[ K = 0.5 m v^2 \]
\[ U = mgh \]
\[ E_{mechanical} = K + U \]

where K is the kinetic energy and U is the potential energy.

\section*{Results}

In the simulations, I set the length to 100 m, both masses to 1 kg, the time
step to 0.01 s and the pendulum in an exact horizontal position. Running
the simulation using the fourth order Runge-Kutta method gave these
results: \\

\begin{tikzpicture}
\begin{axis}[
  title={Evolution of Energy using RK4},
  xlabel=Time (s),
  ylabel=Mechnical Energy (J)]
\addplot table [y=Energy, x=Time]{runge-kutta energy vs time.dat};
\end{axis}
\end{tikzpicture}

This serves as validation for the code. Since the energy is conserved at
2943 J, I know that there are no logical errors. Running the simulation
using Euler's method results in the following: \\

\begin{tikzpicture}
\begin{axis}[
  title={Evolution of Energy using Euler's Method},
  xlabel=Time (s),
  ylabel=Mechnical Energy (J)]
\addplot table [y=Energy, x=Time]{euler energy vs time.dat};
\end{axis}
\end{tikzpicture}

The energy of the system is increasing. It begins at the proper value of
2943 J, but from there it continues to increase indefinitely. The rate of
increase does seem to decrease, but it still displays a fluctuating
behaviour, which is inappropriate for the evolution of the system.

\section*{Discussion}
Euler's method clearly has a major weakness. It cannot adequately represent
the evolution of the double pendulum, let alone of other chaotic systems.
Each iteration has too much error for it even to properly describe the path
of the double pendulum in the first 5 seconds.

\medskip

\section*{References}

\noindent
Weisstein, Eric W. “Double Pendulum.” \textit{Eric Weisstein’s World of Physics},
\indent WolframResearch, 2007, scienceworld.wolfram.com/physics/ \\
\indent DoublePendulum.html.

\noindent
“Animate a Pendulum." \textit{Rosettacode}, May 2019, rosettacode.org/wiki/ \\
\indent Animate\_a\_pendulum.

\noindent
Assencio, Diego. “double-pendulum." \textit{GitHub Inc.}, Dec. 2018, \\
\indent github.com/dassencio/double-pendulum.

\noindent
“Creating a document in Latex." \textit{Overleaf}, 2019, www.overleaf.com/learn \\
\indent /latex/Creating\_a\_document\_in\_LaTeX\#The\_preamble\_of\_a\_document.

\noindent
Nuemann, Erik. “Double Pendulum." \textit{My Physics Lab}, Feb. 2018, \\
\indent www.myphysicslab.com/pendulum/double-pendulum-en.html.

\noindent
Agarwal, Arpit. “Runge-Kutta 4th Order Method to Solve Differential \\
\indent  Equation." \textit{Geeks for Geeks}, https://www.geeksforgeeks.org/runge- \\
\indent kutta-4th-order-method-solve-differential-equation/.


\noindent
“Double Pendulum." \textit{Math 24}, MathJax, www.math24.net/. \\
\indent double-pendulum/

\end{document}
